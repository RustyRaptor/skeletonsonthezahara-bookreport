%  A simple AAU report template.
%  2015-05-08 v. 1.2.0
%  Copyright 2010-2015 by Jesper Kjær Nielsen <jkn@es.aau.dk>
%
%  This is free software: you can redistribute it and/or modify
%  it under the terms of the GNU General Public License as published by
%  the Free Software Foundation, either version 3 of the License, or
%  (at your option) any later version.
%
%  This is distributed in the hope that it will be useful,
%  but WITHOUT ANY WARRANTY; without even the implied warranty of
%  MERCHANTABILITY or FITNESS FOR A PARTICULAR PURPOSE.  See the
%  GNU General Public License for more details.
%
%  You can find the GNU General Public License at <http://www.gnu.org/licenses/>.
%
\input{setup/preamble.tex}% package inclusion and set up of the document
\input{setup/hyphenations.tex}% 
\input{setup/macros.tex}% my new macros

\begin{document}
%frontmatter
\pagestyle{empty} %disable headers and footers
\pagenumbering{roman} %use roman page numbering in the frontmatter
\input{sections/frontpage.tex}
\newpage
\section*{A Brief Summary} 

\subsection*{Introduction}\paragraph{}
On 28 August 1815, the American merchant ship, the \textit{Commerce}, wrecked on the sharp rocks along Cape Bojador, off the coast of modern-day Morocco, missing it's intended course through the canary islands to sell off its valuable cargo, which would, in turn, pay off the debts of its 12 crew-members. `Skeletons On The Zahara` by Dean King is an accurate and insightful story of survival, faith, and the chaotic --- sometimes heart-warming --- collision of two completely different worlds. The story revolves around the journey of the ship's crew and their treacherous journey back home taking unfathomable risks in the process. 


\subsection*{Characters}
\paragraph{} 
James Riley, the Captain and Master of \textit{The Commerce} and it's crew, is the charismatic main protagonist of the story. Riley's strong leadership and social skills not only power the morale of his men but also earn him the respect of everyone around him. These social skills and his ability to quickly learn to understand and speak a little Arabic land him an opportunity with Sidi Hamet, a cunning merchant with many connections throughout Morocco and the surrounding desert settlements, who plays a major role in the plot of the story. 

Aaron Savage was Captain Riley's second-mate. His first-mate, Williams, suffered major injuries from severe sunburns early on, almost losing his eyesight permanently, and thus does not play a major role in the story. Savage, contrary to the beginning on the seas, grows a sour attitude towards Riley and doubts his plans and leadership decisions. This same behavior also ensued a disliking towards him from the Sahrawis and the Arabs. Riley, however, supports Savage throughout the rest of the journey understanding that Savage and his crew were weighed under a rather seemingly ill-fated situation.

Captain Riley's 15-year-old adopted son, Horace, attended the voyage to gain experience as a seaman. Horace is Riley's highest priority on the trip. The captain swore he would not stand before his family again without having the boy with him. Near the end of Riley's journey, Horace is thrown to the ground by Sidi Hamet's brother nearly dying of a concussion which devastated Riley. 

Archibald Robbins, the 22 year old, able seaman, who joined the brig later in New Orleans, was bought by a different Sahrawi and split off from the rest of the crew early on. He has no family back home so his outlook on the situation is different from the other men. Robbins wrote about his own adventure in a journal. 

\subsection*{Plot and Resolution} \paragraph{}
After \textit{The Commerce} ran aground, Riley and his crew were immediately plundered and ransacked by a tribe of Sahrawis. With their lives and freedoms at stake from the Sahrawis, Riley uses one of the older crew members as a decoy and escapes with the rest of the crew within inches of their lives. Using one of the brig's lifeboats they sailed along the coast in dangerous waters. It was at this time the men began to experience the extremes of thirst and hunger that would stick with them throughout the rest of the journey. Drinking their own urine and eating minuscule rations of salt pork they made it on a seeming safe land where they were immediately found by a more civilized tribe of Sahrawis known as the Oulad Bou Sbaa. A member of the tribe takes ownership of them as slaves. They are stripped of all their belongings including their clothing and now travel along the tribe. After several weeks of starvation, travel injury, and sunburn, Oulad Bou Sbaa came into contact with a tradesman, Sidi Hamet.

Riley had learned to speak and understand enough Arabic to communicate with the merchant so he explained his situation to him. Sidi Hamet sympathized with Riley and swore to him that he will be back safely with his family one day. A deal was made between Riley and Hamet that if Hamet buys Riley and some of his crewmen, Riley will get a '\textit{friend of his}' in Swearah to repay Hamet for the costs of the purchase, the costs of provisions, an additional fee, and two double-barrelled muskets. If Riley is unable to hold his side of the deal Hamet would cut his throat and sell his crew for as much as possible to the south. The truth was Riley did not have any friends he knew in Swearah, yet he managed to convince Hamet, a cunning salesman, that he was speaking the truth.

Sidi Hamet saw this deal as an opportunity to be able to pay off his debt to his father-in-law, Sheikh Ali. Ali is the leader of a large tribe and has considerable power. Riley and Hamet develop a strong friendship through their travels to the northern town of Swearah. After a difficult, risky, and seemingly endless journey, Hamet gives Riley the names of all the consuls in Swearah, asking him to point out which one he knew. Riley pointed to William Wilshire because it was a western sounding name. Riley carefully wrote a letter to Wilshire explaining the situation and pleading for his help. Hamet and a companion took off to Swearah to deliver the letter, leaving the crew in a camp with his brother. Fortunately for the men of \textit{The Commerce}, William knew of the sufferings of western slaves on the Sahara and obtained the money needed to pay Sidi Hamet.


During this time, complications did arise. Sheikh Ali arrived and found out about the Christians that Hamet had in his possession. Sheikh Ali plotted to take the slaves for himself but after a long struggle and a trip to another town --- thanks to Riley's social abilities --- Hamet and the crew were triumphant. Riley and his men made it to Swearah and were given a warm welcome by the British consul William Wilshire. Wilshire also sent out for the remaining of the crew. The majority of the men were reunited. Sidi Hamet also went in search for more crew members, unfortunately, he was killed protecting the men from a group of thieves who demanded that he give them the crew members (Those crew members did make it home safely, however). 

The rest is history. Riley goes on to oppose slavery in the United States, citing that the very word fills him with discomfort. One retelling of the story ends up inspiring U.S. President Abraham Lincoln's campaign and policies against slavery. 

\subsection*{About The Insight and Knowledge The Book Provides} \paragraph*{}
The primary feature that distinguishes Skeletons On The Zahara from previous reiterations of Captain Riley's story is the extensive research and elaboration done on the author's part. Dean King traveled to the Sahara before writing the book and attempted to follow through Riley's path. This gave King a plethora of insight into the journey never before seen. He points out certain areas where Riley was likely mistaken about his whereabouts. The book also cross-references information from other accounts of westerners being sold into slavery in the area which confirm many things that were said to have happened. We also get several explanations for the way the Sahrawis behaved based on their culture and religion. All of this in addition to countless other '\textit{fun facts}' throughout the story jumping to past and future events that are connected.  Skeletons On The Zahara is truly the ultimate retelling of the story. 

\subsection*{Thoughts and Conclusion}

After reading Skeletons On The Zahara it is indeed, very clear, that it is not a mere retelling of an old tale. The book presents the story in a way that is helpful and interesting to the modern reader. The story of Captain Riley had me faced with a variety of emotions. I can safely say I have a much higher appreciation of the availability of food, water, and financial support we have today as opposed to what we had 200 years ago. There is also something to be said about the importance of people with strong social and leadership skill judging by the events in the story. \newline

\begin{quote}
"The crew of the \textit{Commerce} seem to have been designed to suffer themselves that the world, through them, might learn "\newline \begin{center}
--- Archibald Robbins, \textit{A Journal Comprising an Account of the Loss of the Brig Commerce}
\end{center}

\end{quote}


\end{document}
